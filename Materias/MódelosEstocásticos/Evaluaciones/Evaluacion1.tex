\documentclass{article}
\usepackage{amsmath} 
\usepackage{amsthm}
\usepackage{amsfonts}
\usepackage{enumitem}
\usepackage{tcolorbox}
\usepackage[a4paper, bottom=3cm]{geometry}

\title{Stochastic Models - Homework 1}
\author{César Pérez, Luciana Hernández, Dulce María San Martín}
\date{September 2025}

\begin{document}

\theoremstyle{definition}
\newtheorem{solutions}{Solution}
\maketitle

\section*{Solutions}
    \begin{minipage}{\linewidth}
        \begin{solutions}
            We want the number of women that are single and young, if we denote young = $J$, men = $H$, married = $C$, then we want $\#(J \cap H^{c} \cap C^c)$ that is:
            \[
                \#J - \#(J \cap H) - \#(J \cap C) + \#(J \cap H \cap C)
            \]
            Since we are taking the young people, eliminating if they are men or married, since there are young married men, and we eliminated the young married and the young men we eliminated two times the intersection of the 3, hence we have to add the intersection of these 3 (to avoid double eliminating young people). Then we have:
            \[
                3000 - 1320 - 1400 + 600 = 880
            \]
        \end{solutions}
        \medskip
    \end{minipage}
    \begin{minipage}{\linewidth}
        \begin{solutions}
            For $P$ to be a probability measure, the following statements have to be true: $P(A) \geq 0, P(\Omega) = 1$ and the additive property, which already is assumed to be true in the problem.
            Also we have that 
            \[
                P(\{k\}) = \sum_{k \in A}P(\{k\}) = P(k_1) + P(k_2) + ..., 
            \]
            where $k_1, k_2, ... \in A$.
                \begin{enumerate}
                    \item $P(A) \geq 0$. We know that since $\Omega = \mathbb{N}, k \geq 1$, and $P(\{k\}) = \frac{1}{k(k+1)}$, we always have positive numbers, whose sum is always positive. Hence $P(A) \geq 0, \forall A \in \mathcal{F}$
                    \item $P(\Omega) = 1$.
                    We have that 
                    \[
                        P(\Omega) = \sum_{k \in \Omega} P(\{k\}).
                    \]
                    Since $\Omega = \mathbb{N}$, we can write 
                    \[
                        P(\Omega) = P(1) + P(2) + \cdots 
                        = \sum_{i = 1}^{\infty} \frac{1}{i(i + 1)}.
                    \]
                    This is a telescopic series that converges to $1$; hence, $P(\Omega) = 1$.
                \end{enumerate}
            Since we proved Kolmogorov's axiom 1 and 2 hold, and axiom 3 (additive property) is assumed in the problem, $P$ is a probability measure. 
        \end{solutions}
    \end{minipage}
    \begin{minipage}{\linewidth}
        \begin{solutions}
            For P to be a probability, the following statements have to be true $P(A) \geq 0, P(\Omega) = 1$, and the additive property: if $A_ 1, A_2, A_3, ...$ are pairwise disjoint, then: 
                \[  
                    P(\bigcup_{i = 1}^{\infty} A_i) = \sum_{i = 1}^{\infty} P(A_i)
                \]
                \begin{enumerate}
                    \item $P(A) \geq 0$ We have: 
                    \[
                        P(A) =\int_{A}f(x)dx = \int_{\mathbb{R}}1_A(x)f(x)dx
                    \]
                    where $1_A(x)$ is the indicator function of $A$, now, if $x \notin A$ the integral becomes 0, and if $x \in A$ we have the integral of $f(x)$ and since $f(x)$ is always positive, the integral is always positive. Hence $P(A) \geq 0, \forall A \in \mathcal{F}$.
                    \item $P(\Omega) = 1$ We have: 
                    \[
                        P(\Omega) = \int_{\mathbb{R}} 1_\Omega(x) f(x) dx
                    \]
                    But since $\Omega = \mathbb{R}$ and we are integrating over $\mathbb{R}$ then the indicator function of $\Omega$ is always equal to 1, then we are left with:
                    \[
                        \int_{-\infty}^{0} f(x)dx + \int_{0}^{1}f(x)dx + \int_{1}^{\infty} f(x)dx
                    \]
                    By definition of our function $f(x)$ we have that the first and last integral are equal to 0, hence we are left with:
                    \[
                        \int_{0}^{1} 2xdx = x^2\Big|_0^1 = 1
                    \]
                    Hence $P(\Omega) = 1$.
                    \item Additive property. We have that:
                    \[
                        P(\bigcup_{i = 1}^{\infty}A_i) =\int_{\bigcup_{i = 1}^{\infty}A_i} f(x)dx = \int_{\mathbb{R}} 1_{\bigcup_{i = 1}^{\infty} A_i}(x) f(x)dx 
                    \]
                    Since $A_1, A_2, A_3, ...$ are disjoint events we have that:
                    \[
                        1_{\bigcup_{i = 1}^{\infty} A_i}(x) = \sum_{i = 1}^{\infty}1_{A_i}(x)
                    \]
                    Then:
                    \[
                        \int_{\mathbb{R}} 1_{\bigcup_{i = 1}^{\infty} A_i}(x) f(x)dx = \int_{\mathbb{R}} \sum_{i = 1}^{\infty}1_{A_i}(x)f(x)dx =  \sum_{i = 1}^{\infty} \int_{\mathbb{R}} 1_{A_i}(x)f(x)dx = \sum_{i = 1}^{\infty} P(A_i) 
                    \]
                    Hence the additive property holds.
                    It is worth mentioning that we can interchange sum and integral because they are positive always. 
                \end{enumerate}
                Since all Kolmogorov's axioms hold, $P$ is a probability measure. 
        \end{solutions}
    \end{minipage}
    \begin{minipage}{\linewidth}
        \begin{solutions}
            We want to proof: 
            \[
                P(A \cup B \cup C) = P(A) + P(B) + P(C) - P(A \cap B) - P(A \cap C) - P(B \cap C) + P(A \cap B \cap C)
            \]
            \begin{proof}
                We can take the first side of the equation and start operating with the previously proved propositions of axiomatic probability: 
                \begin{gather*}
                    P(A \cup B \cup C) = P((A \cup B) \cup C) = P(A \cup B) + P(C) - P((A \cup B) \cap C)\\ 
                    = P(A) + P(B) - P(A \cap B) + P(C) - P((A \cup B) \cap C)\\
                    = P(A) + P(B) - P(A \cap B) + P(C) - P((A \cap C) \cup (B \cap C))\\
                    = P(A) + P(B) - P(A \cap B) + P(C) - (P(A \cap C) + P(B \cap C) - P((A \cap B) \cap (A \cap C)))\\
                    = P(A) + P(B) - P(A \cap B) + P(C) - P(A \cap C) - P(B \cap C) + P(A \cap B \cap C)\\
                    = P(A) + P(B) + P(C)- P(A \cap B) - P(A \cap C) - P(B \cap C) + P(A \cap B \cap C)
                \end{gather*}
            \end{proof}
            \medskip
        \end{solutions}
    \end{minipage}
    \begin{minipage}{\linewidth}
        \begin{solutions}
            We have that $P(A) = 0.7, P(B) = 0.9$, and have to calculate the maximum value $P(A \cup B) - P(A \cap B)$ can take. 
            We have that:
            \[
                P(A \cup B) = P(A) + P(B) - P(A \cap B) = 0.7 + 0.9 - P(A \cap B)
            \]
            Since we have that $0 \leq P(A \cup B) \leq 1$, we have that $0.7 + 0.9 - P(A \cap B) \leq 1 \implies 0.6 \leq P(A \cap B)$, also, applying the property of inequalities of probabilities, we know that:
            \[
                P(A \cap B) \leq (\min{\{P(A), P(B)\}} = 0.7)
            \]
            Then to maximize the result we can take the minimum value of the two we know less or equal than $P(A \cap B)$ which is $\min{\{0.6, 0.7\}} = 0.6$. Hence the maximum value of the expression is:
            \[
                P(A \cup B) - P(A \cap B) = 1 - 0.6 = 0.4
            \]
            \medskip
        \end{solutions}
    \end{minipage}
    \begin{minipage}{\linewidth}
        \begin{solutions}
            If a die is rolled 4 times, what is the probability of getting 6 at least once: since we have the experiment of rolling a die and we are repeating it 4 times, each with a classical probability of $\frac{1}{6}$, we can calculate the probability of the complement and use it to get the probability, let $A$ be "the result of the die is 6":
            \[
                P(A^c) = \frac{5}{6} 
            \]
            Now since we are repeating the experiment 4 times we have: 
            \[
                \frac{5}{6}  \cdot \frac{5}{6} \cdot \frac{5}{6} \cdot \frac{5}{6} = (\frac{5}{6})^{4} = \frac{625}{1296}
            \]
            To calculate the probability of the original event we can use the property of the probability of the complement of an event:
            \[
                1 - \frac{625}{1296} = \frac{671}{1296}
            \]
            \medskip
        \end{solutions}   
    \end{minipage}
    \begin{minipage}{\linewidth}
        \hspace*{-\parindent}
        \begin{solutions}
            If $P(A) = \frac{1}{3} \land P(B^c) = \frac{1}{4}$ is it possible for $A$ and $B$ to be disjoint? First we calculate the probability of $B$ by using the property of the complement: $(B^c)^c = B$, since $P(B^c) = \frac{1}{4}, P(B) = 1 - \frac{1}{4} = \frac{3}{4}$. Then we can check if they are disjoint by calculating the probability of the union.
            \[
                P(A \cup B) = P(A) + P(B) - P(A \cap B) = \frac{1}{3} + \frac{3}{4} - P(A \cap B)= \frac{13}{12} - P(A \cap B) 
            \]
            They can't be disjoint since $P(A \cup B) \leq 1$ and if $A$ and $B$ are disjoint sets, then
            \[
                P(A \cap B) = P(\emptyset) = P(0) \implies P(A \cup B) = \frac{13}{12} - 0 = \frac{13}{12}
            \]
            then $P(A \cup B) > 1$, but this can't happen.
            \medskip           
        \end{solutions}
        \begin{solutions}
        A password is formed with the letters ${A, B, C, D, E, F}$
            \begin{enumerate}[label=\alph*)]
                \item If $A$ and $B$ are together, it is a problem with order and without replacement. We use 
                \[
                    \frac{n!}{(n-k)!}
                \]
                Then if $n=5$ and $k=5$, n! = 5! = 120               
                Since $A$ and $B$ can be ordered either $AB$ or $BA$, we multiply by $2! = 2$
                 \[
                    120 * 2 = 240
                 \]
                Therefore, we can create \textbf{240} passwords where $A$ and $B$ are together.
                
                \item If $A$ is before $B$, we can compute the password 
                \[
                    6! = 720 
                \] 
                ways; However, A is before B in half of these experiment, then we divide by $2!$
                \[
                    \frac{720}{2} = 360
                \]
                Therefore, we can create \textbf{360} passwords where $A$ is before $B$.
                
                \item A is before of B and B is before of C, we can compute the password
                \[
                    6! = 720
                \]
                Taking into account our restriction, there are 3 different ways to arrange the letters: 
                \[
                    3! = 6
                \]
                Therefore:
                \[
                    \frac{6!}{3!} = 120
                \]
                Hence, we can create \textbf{120} passwords where $A$ is before $B$, and $B$ is before $C$.
                \item A and B are together and C and D are also together
                There are 
                \[
                    4! = 24
                \]
                since we have 4 "blocks" we can make out of. And then
                \[
                    4! \cdot 2! \cdot 2! = 96
                \]
                Therefore, we can create \textbf{96} passwords where $A$ and $B$ are together and $C$ and $D$ are together.
            \end{enumerate}
            \medskip
        \end{solutions}
    \end{minipage}
        \begin{solutions}
            Three groups are considered, each consisting of N students. A total of 3 students are selected from these 3 groups.
            \begin{enumerate}[label=\alph*)]
                \item How many different possible choices can be made?
                We got 3 groups with $N$ students each, then $n = 3N$
                \[
                    \binom{n}{k} = \frac{n!}{(n-k)! \cdot k!}
                \]
                Therefore;
                \[
                    \binom{3N}{3} = \frac{(3N)!}{(3N-3)! \cdot 3!} = \frac{\mathbf{(3N)\cdot(3N-1)\cdot(3N-2)}}{\mathbf{6}}
                \]

                \item How many possible choices can be made if all 3 students are asked to belong to the same group?
                \[
                    \binom{n}{3} = \frac{n!}{(n-3)!3!}= \frac{\mathbf{(n)(n-1)(n-2)}}{\mathbf{6}}
                \]

                \item How many possible choices can be made if 2 of the 3 students are asked to belong to the same group, and the other to belong to a different group?
                If two are in the same group:
                \[
                    \binom{n}{2} = \frac{n!}{(n-2)! \cdot 2!} = \frac{\mathbf{(n) \cdot (n-1)}}{\mathbf{2}}
                \]
                If one is in a different group:
                \[
                    \binom{n}{2} (3) \cdot (2n) = 3 \frac{(n) \cdot (n - 1)}{2} (2n) =
                    3 \cdot (n) \cdot (n-1) \cdot (n) = 3n^2(n-1) = \mathbf{3n^3 - 3n^2}
                \]
            \end{enumerate}
        \end{solutions}
        \begin{solutions}
            A 5-card poker hand is said to be a full house if it consists of 3 cards of the same rank and the other two cards have the same rank. What is the classical probability that a player’s hand is a full house?
            \noindent We know that:
            \begin{enumerate}
                \item There are 52 cards in a deck
                \item There are a total of 13 different numbers 
                \item There are 4 suits in a deck
            \end{enumerate}
            Also, a full house is a poker hand with three of a kind and a pair, hence if the three first are the same:
            \[
                \frac{4!}{3! \cdot (4 - 3)!} = 4
            \]
            Also, if the other two are the same:
            \[
                \frac{4!}{2! \cdot (4 - 2)!} = 6
            \]
            Then the cardinality of A:
            \[
                \#A = 13 \cdot 4 \cdot 12 \cdot 6
            \]
            We know that, since we are looking for ways of acommodating 52 cards in a hand of 5 we have: 
            \[
                \Omega = \binom{52}{5} 
            \]Therefore, the classical probability of getting a full house is:
            \[
                \mathbf{P(A)} = \frac{\mathbf{13 \cdot 4 \cdot 12 \cdot 6}}{\mathbf{\binom{52}{5}}} = \frac{\mathbf{3744}}{\mathbf{\binom{52}{5}}}
            \]
            \medskip
        \end{solutions}
        \begin{solutions}
            A teacher gives a list of 10 exercises to his students. It is known that the final exam consists of a random selection of 5 of them. If a student managed to find the answer to 7 of the 10 exercises, what is the probability that he/she will correctly answer at least 4 of the final problems? \par
            \bigskip
            \noindent In order to know $\Omega$ we need to know how many ways there are to arrange the 10 questions (Without Order). 
            \[
                \binom{10}{5} = \frac{10!}{(10-5)! \cdot 5!} = 252
            \]
            Since there are 7 known and 3 unknown answers; 
            \[
                C_{5}^{7}
            \]
            Plus, we consider that the 3 unknown answers will not appear in the exam, then:
            \[
                C_{5}^{7} \cdot C_{0}^{3} = 21
            \]Furthermore, we got to take into account the case when the student doesn't know an answer:
            \[
                C_{4}^{7} \cdot C_{1}^{3} = 105
            \]
            Finally, we have that:
            \[
                \#A = (C_{5}^{7} \cdot C_{0}^{3}) + (C_{4}^{7} \cdot C_{1}^{3}) = \mathbf{126}
            \]
            Thus
            \[
                P(A) = \frac{\#A}{\#\Omega} = \frac{126}{252} = \mathbf{0.5}
            \]
            \medskip
        \end{solutions}
        \begin{solutions}
            From a group of 8 men and 6 women, a committee of 3 men and 3 women is formed. In how many ways can the committee be formed if a man and a woman are not willing to be part of the committee at the same time? \par
            \[ 
                \#\Omega = \binom{8}{3} \cdot \binom{6}{3} = \frac{8!}{(8 - 3)! \cdot 3!} \cdot \frac{6!}{(6 - 3)! \cdot 3!} = 1120
            \]
            This is since we are looking for all the possible ways of acommodating 8 men in a group of 3 and 6 women in a group of 3 
            Then to calculate the cardinality ways for the committee to be formed if a man and a woman are not willing to be a part of the committee at the same time we can assume they alredy are in the committe:
            \[ 
                \binom{7}{2} \cdot \binom{5}{2} = \frac{7!}{(7 - 2)! \cdot 2!} \cdot \frac{5!}{(5 - 2)! \cdot 2!} = 210
            \]
            Then the possible ways are just all the possible ways without restrictions minus the possible ways when they alredy are in the committee:
            \[ 
                1120 - 210 = 910
            \]
            \noindent 
            \bigskip
            \bigskip
        \end{solutions}
    \begin{samepage}
        \begin{solutions}
            A password is formed using the following letters:
            \begin{enumerate}[label=\alph*)]
                \item Probability
                \item Statistics
            \end{enumerate} 
            How many passwords can be made? \par
            \bigskip
            \noindent a) The word \textit{"probability"} has a total of $11$ letters, of which the number of times \textit{"b"} is repeated is 2, same as \textit{"i"}, the rest of the letters are only repeated 1 time, hence we are in the case with order and without replacement, therefore:
            \[
                \frac{11!}{1! \cdot 1! \cdot 1! \cdot 2! \cdot 1! \cdot 2! \cdot 1! \cdot 1! \cdot 1!} = \frac{11!}{2! \cdot 2!} = \mathbf{9, 979, 200}
            \]
            b) The word \textit{"statistics"} has a total of $10$ letters, of which the number of times \textit{"s"} and \textit{"t"} are repeated 3 times , and \textit{"i"} 2 times, the rest of the letters are only repeated 1 time, hence we are in the case with order and without replacement, therefore;
            \[
                \frac{10!}{3! \cdot 3! \cdot 2! \cdot 1! \cdot 1!} = \frac{10!}{3! \cdot 3! \cdot 2!} = \mathbf{50, 400}
            \]
            \medskip
        \end{solutions}
    \end{samepage}
        \begin{solutions}
            4 red balls, 8 blue balls and 5 green balls are randomly placed in an array.
            \begin{enumerate}[label=\alph*)]
                \item What is the classical probability that none of the first 5 balls are blue?
                \item What is the classical probability that the red balls end up together?
            \end{enumerate}
            \medskip
            \noindent a) We can calculate this probability by calculating the probabilities of the ball not being blue the first five events, that is, since we have 17 total objects and 8 of them are blue:
            \[ 
                \frac{9}{17} \cdot \frac{8}{16} \cdot \frac{7}{15} \cdot \frac{6}{14} \cdot \frac{5}{13} = \frac{63}{3094}
            \]
            \noindent b) We can calculate this probability by making the 4 balls a single "block" remembering that the balls are indistinguishable, and we also need the cardinality of $\Omega$:
            \[ 
                \#\Omega = \frac{17!}{4! \cdot 8! \cdot 5!}, \# A = \frac{14!}{5! \cdot 8!} 
            \]
            Then we can calculate the probability using our definition of classical probability: $P(A) = \frac{\#\Omega}{\# A}$
            \[ 
                = \frac{14!}{8! \cdot 5!} \cdot \frac{4! \cdot 5! \cdot 8!}{17!} = \frac{14! \cdot 4! \cdot 8! \cdot 5!}{17! \cdot 5! \cdot 8!} = \frac{4!}{17 \cdot 16 \cdot 15} = \frac{1}{170}
            \]
        \end{solutions}
\end{document}

