\documentclass[12pt,a4paper]{article}

% Incluye preámbulo
\usepackage[most]{tcolorbox}
\usepackage{amsmath, amsthm, amssymb}
\usepackage[margin=2.5cm]{geometry}

\tcbuselibrary{theorems}

\newtcbtheorem[number within=section]{definicion}{Definición}%
{colback=blue!5!white, colframe=blue!75!black, fonttitle=\bfseries}{def}
\newtcbtheorem[number within=section]{teorema}{Teorema}%
{colback=green!5!white, colframe=green!75!black, fonttitle=\bfseries}{teo}
\newtcbtheorem[number within=section]{proposicion}{Proposición}%
{colback=yellow!10!white, colframe=yellow!50!black, fonttitle=\bfseries}{prop}
\newtcbtheorem[number within=section]{notacion}{Notación}%
{colback=blue!5!white, colframe=red!75!black, fonttitle=\bfseries}{def}

\newcommand{\N}{\mathbb{N}}
\newcommand{\Z}{\mathbb{Z}}
\newcommand{\Q}{\mathbb{Q}}
\newcommand{\R}{\mathbb{R}}
\newcommand{\C}{\mathbb{C}}


% Información del documento
\title{Título de tus notas aquí}
\author{Cesar Perez Amador}
\date{\today}

\begin{document}

\maketitle

\section{Primera sección a cambiar}

\begin{definicion}{Ejemplo de definición}{ej1}
Un número natural es un elemento de $\N = \{1, 2, 3, \dots\}$.
\end{definicion}

\begin{teorema}{Ejemplo de teorema}{teo1}
Sea $n \in \N$. Si $n$ es par, entonces $n^2$ también es par.
\end{teorema}

\begin{proposicion}{Ejemplo de proposición}{prop1}
Si $a,b \in \R$ son positivos, entonces $a+b>0$.
\end{proposicion}

\section{Segunda sección a cambiar}

\begin{definicion}{Otra definición}{ej2}
Un número primo es un número natural mayor que 1 que solo tiene dos divisores positivos: 1 y él mismo.
\end{definicion}

\begin{teorema}{Otro teorema}{teo2}
Todo número primo mayor que 2 es impar.
\end{teorema}

\begin{proposicion}{Otra proposición}{prop2}
Si $x \in \R$ y $x^2 = 0$, entonces $x=0$.
\end{proposicion}

\end{document}
