\documentclass[12pt,a4paper]{article}

% Incluye preámbulo
\usepackage[most]{tcolorbox}
\usepackage{amsmath, amsthm, amssymb}
\usepackage[margin=2.5cm]{geometry}

\tcbuselibrary{theorems}

\newtcbtheorem[number within=section]{definicion}{Definición}%
{colback=blue!5!white, colframe=blue!75!black, fonttitle=\bfseries}{def}
\newtcbtheorem[number within=section]{teorema}{Teorema}%
{colback=green!5!white, colframe=green!75!black, fonttitle=\bfseries}{teo}
\newtcbtheorem[number within=section]{proposicion}{Proposición}%
{colback=yellow!10!white, colframe=yellow!50!black, fonttitle=\bfseries}{prop}
\newtcbtheorem[number within=section]{notacion}{Notación}%
{colback=blue!5!white, colframe=red!75!black, fonttitle=\bfseries}{def}

\newcommand{\N}{\mathbb{N}}
\newcommand{\Z}{\mathbb{Z}}
\newcommand{\Q}{\mathbb{Q}}
\newcommand{\R}{\mathbb{R}}
\newcommand{\C}{\mathbb{C}}


% Información del documento
\title{\Huge Sucesiones}
\author{Cálculo 2 - MAT2012 - César Pérez}
\date{Otoño 2025}

\begin{document}

\maketitle

\section{Definiciones}

\begin{definicion}{Función sobreyectiva}{def1}
    Sea $f: A \to B$. Decimos que $f$ es sobreyectiva si 
    \[
        \forall b \in B, \exists a \in A: f(a) = b
    \]
\end{definicion}

\begin{definicion}{Sucesión}{def2}
    Una sucesión es una función sobre los naturales de la forma $f: \N \to \R$ 
\end{definicion}

\begin{notacion}{}{ntc1}
    Denotamos las sucesiones de las siguientes formas:
    \[ 
        a_n \equiv \{a_n\} \equiv \{a_n\}_{n = 1}^{\infty} \equiv f(n) \equiv (a_n)
    \]
\end{notacion}
    
\begin{definicion}{Límite de una sucesión}{def3}
    Decimos que la sucesión $a_n$ tiene límite L, denotado por
    \[ 
        \lim_{n \to \infty} a_n = L
    \]
    si se da que 
    \[ 
        \forall \varepsilon > 0, \exists N \in \N: \text{para } n > N \text{ se cumple } \lvert a_n - L \rvert < \varepsilon
    \]
\end{definicion}

\begin{definicion}{Sucesiones de Cauchy}{def4}
    Decimos que la sucesión $a_n$ tiene límite L, denotado por
    \[ 
        \lim_{n \to \infty} a_n = L
    \]
    si se da que 
    \[ 
        \forall \varepsilon > 0, \exists N \in \N: \lvert a_n - a_m \rvert < \varepsilon, \forall m,n > N 
    \]
\end{definicion}

\section{Teoremas}

\begin{teorema}{Sucesión de funciones}{thm1}
    Sea la sucesión $a_n$, si:
    \[ 
        \lim_{n \to \infty} f(x) = L \text{ y } f(n) = a_n, \forall n \in \N
    \]
    entonces
    \[ 
        \lim_{n \to \infty} a_n = L
    \]
\end{teorema}

\begin{teorema}{Álgebra de sucesiones}{thm2}
    Sea $\{a_n\}$ y $\{b_n\}$ sucesiones con límites existentes, respectivamente cada límite vale $L_1$ y $L_2$ y $c\in\mathbb{R}$. Se cumple:
    \[
    \begin{aligned}
    &\text{Suma/Resta:} & \lim_{n \to \infty} (a_n \pm b_n) &= \lim_{n \to \infty} a_n \pm \lim_{n \to \infty} b_n,\\
    &\text{Producto por escalar:} & \lim_{n \to \infty} (c\,a_n) &= c\,\lim_{n \to \infty} a_n,\\
    &\text{Producto:} & \lim_{n \to \infty} (a_n b_n) &= (\lim_{n \to \infty} a_n)(\lim_{n \to \infty} b_n),\\
    &\text{Cociente:} & \lim_{n \to \infty} \frac{a_n}{b_n} &= \frac{\lim\limits_{n \to \infty} a_n}{\lim\limits_{n \to \infty} b_n}, \quad \lim_{n\to\infty} b_n \neq 0. \\
    &\text{Potencia:} & \lim_{n \to \infty} (a_n)^m &= L_1^m,\\
    &\text{Continuidad ($f$ continua): } & \lim_{n \to \infty}f(a_n) = f(L_1) \\
    \end{aligned}
    \]
\end{teorema}

\begin{teorema}{Criterio del punto fijo}{thm3}
    Sea $(a_n)$ una sucesión, y supongamos que $\lim_{n \to \infty} = L$. Se cumple que:
    \[ 
        \lim_{n \to \infty} a_n = \lim_{n \to \infty} a_{n + 1} = L
    \]
    En particular si $a_{n + 1} = f(a_n)$ con $f$ continua, se da que $L = f(L)$
\end{teorema}

\begin{teorema}{Compresión de sucesiones}{thm4}
    Si $\lim_{n \to \infty} a_n = L = \lim_{n \to \infty} c_n$ y $a_n \leq b_n \leq c_n$ entonces
    \[ 
        \lim_{n \to \infty} b_n = L
    \]
\end{teorema}

\begin{teorema}{Convergencia del absoluto}{thm5}
    Si $\lim_{n \to \infty} \lvert a_n \rvert = L$ entonces $\lim_{n \to \infty} a_n = L
\end{teorema}

\end{document}
