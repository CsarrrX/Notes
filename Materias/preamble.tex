\usepackage[spanish,es-noshorthands]{babel}

\usepackage{xcolor}
\usepackage[most]{tcolorbox}
\usepackage{amsmath, amsthm, amssymb}
\usepackage[margin=2.5cm]{geometry}

\tcbuselibrary{theorems, breakable}

% Asegurar texto negro dentro de cajas y evitar "color leak"
\tcbset{coltext=black}
\AtBeginDocument{\color{black}}

% Entornos tipo teorema: hacerlos breakable y dar prefijos únicos
\newtcbtheorem[number within=section]{definicion}{Definición}%
{enhanced, breakable, colback=blue!5!white, colframe=blue!75!black, fonttitle=\bfseries}{def}
\newtcbtheorem[number within=section]{teorema}{Teorema}%
{enhanced, breakable, colback=green!5!white, colframe=green!75!black, fonttitle=\bfseries}{teo}
\newtcbtheorem[number within=section]{proposicion}{Proposición}%
{enhanced, breakable, colback=yellow!10!white, colframe=yellow!50!black, fonttitle=\bfseries}{prop}
\newtcbtheorem[number within=section]{notacion}{Notación}%
{enhanced, breakable, colback=blue!5!white, colframe=red!75!black, fonttitle=\bfseries}{ntc}

\newtcolorbox{nota}{
    enhanced,
    breakable,
    colback=white,
    colframe=white,
    coltitle=black,
    fonttitle=\bfseries,
    title=Nota:,
    attach title to upper={\ },
    borderline west={2pt}{0pt}{blue!30},
    left=6pt,
    right=6pt,
    top=6pt,
    bottom=6pt
}

\newcommand{\N}{\mathbb{N}}
\newcommand{\Z}{\mathbb{Z}}
\newcommand{\Q}{\mathbb{Q}}
\newcommand{\R}{\mathbb{R}}
\newcommand{\C}{\mathbb{C}}

% Carga hyperref al final
\usepackage{hyperref}
\hypersetup{
  colorlinks=true,
  linkcolor=blue,
  urlcolor=blue,
  citecolor=magenta
}
