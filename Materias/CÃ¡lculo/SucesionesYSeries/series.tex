\documentclass[12pt,a4paper]{article}

% Incluye preámbulo
\usepackage[most]{tcolorbox}
\usepackage{amsmath, amsthm, amssymb}

\geometry{margin=2.5cm}

\tcbuselibrary{theorems}

\newtcbtheorem[number within=section]{definicion}{Definición}%
{colback=blue!5!white, colframe=blue!75!black, fonttitle=\bfseries}{def}
\newtcbtheorem[number within=section]{teorema}{Teorema}%
{colback=green!5!white, colframe=green!75!black, fonttitle=\bfseries}{teo}
\newtcbtheorem[number within=section]{proposicion}{Proposición}%
{colback=yellow!10!white, colframe=yellow!50!black, fonttitle=\bfseries}{prop}

\newcommand{\N}{\mathbb{N}}
\newcommand{\Z}{\mathbb{Z}}
\newcommand{\Q}{\mathbb{Q}}
\newcommand{\R}{\mathbb{R}}
\newcommand{\C}{\mathbb{C}}


% Información del documento
\title{\Huge Series}
\author{Cesar Perez Amador}
\date{\today}

\begin{document}

\maketitle

\section{Definiciones}
Ya definidas las sucesiones podemos empezar a trabajar con series: 

\begin{definicion}{Series}{def1}
    Una serie es la suma de los terminos de una sucesión $(a_n)$, es decir una serie esta dada por 
    \[ 
        \sum_{n = 1}^{\infty} a_n
    \]
\end{definicion}

Como en sucesiones, tenemos un "problema" que resolver, en paralelo con sucesiones, tenemos que determinar la convergencia de las series:

\begin{definicion}{Convergencia de series}{def2}
    Decimos que un serie es convergente, denotado por:
    \[ 
        \lim_{n \to \infty} \sum_{i = 1}^{n} a_i = L
    \]
    Si la sucesion $S_n = \sum_{n = 1}^n a_i$ de sumas parciales es convergente, es decir $\lim_{n \to \infty} S_n$ existe. 
\end{definicion}

Nota: $S_n = \{a_1, a_1 + a_2, a_1 + a_2 + a_3, \cdots \} 

\section{Series conocidas}


\end{document}
