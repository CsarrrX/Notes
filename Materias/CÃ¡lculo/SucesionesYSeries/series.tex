\section{\Large Series}

\subsection{Definiciones}

Ya definidas las sucesiones podemos empezar a trabajar con series: 

\begin{definicion}{Series}{def1}
    Una serie es la suma de los terminos de una sucesión $(a_n)$, es decir una serie esta dada por 
    \[ 
        \sum_{n = 1}^{\infty} a_n
    \]
\end{definicion}

Como en sucesiones, tenemos un "problema" que resolver, en paralelo con sucesiones, tenemos que determinar la convergencia de las series:

\begin{definicion}{Convergencia de series}{def2}
    Decimos que un serie es convergente, denotado por:
    \[ 
        \lim_{n \to \infty} \sum_{i = 1}^{n} a_i = L
    \]
    Si la sucesion $S_n = \sum_{n = 1}^n a_i$ de sumas parciales es convergente, es decir $\lim_{n \to \infty} S_n$ existe. 
\end{definicion}

Nota: $S_n = \{a_1, a_1 + a_2, a_1 + a_2 + a_3, \cdots \}$

\subsection{Series conocidas}

Nuestra primera forma de criterio de convergencia son series conocidas:

\begin{definicion}{Serie armónica}{def3}
    Esta es la primera serie conocida, esta dada por:
    \[ 
        \sum_{n = 1}^{\infty} \frac{1}{n} 
    \]
    esta serie es \textbf{divergente}
\end{definicion}

\begin{definicion}{Serie geometrica}{def4}
    Esta serie esta dada por un número fijo por una razón elevada a la n - 1:
    \[ 
        \sum_{n = 1}^{\infty} ar^{n - 1} 
    \]
    en donde $a$ es el número fijo y $r$ es la razón
\end{definicion}

Sin embargo la definición de la serie, no nos determina su convergencia, para eso tenemos:

\begin{teorema}{Convergencia de la serie geometrica}{thm1}
    La serie geometrica es: 
    \begin{enumerate}
        \item Convergente si $|r| < 1$
        \item Divergente si $|r| > 1$
    \end{enumerate}
\end{teorema}

\begin{definicion}{Serie telescopica}{def5}
    Esta serie es la resta de dos terminos de la solución:
    \[ 
        \sum_{n = 1}^{\infty} a_n - a_{n + k} 
    \]
    Generalmente converge al primer termino, ya que se cancelan los terminos
\end{definicion} 

\subsection{Criterios de la convergencia}
\begin{teorema}{Criterio necesario}{crit1}
    Si la serie 
    \[
        \sum_{n=1}^{\infty} a_n
    \]
    es convergente, entonces 
    \[
        \lim_{n \to \infty} a_n = 0.
    \]

    \textbf{Nota:} Usamos la contrarrecíproca:  
    Si 
    \[
        \lim_{n \to \infty} a_n \neq 0,
    \]
    entonces la serie 
    \(\sum_{n=1}^{\infty} a_n\) no converge.
\end{teorema}

\begin{teorema}{Prueba de la integral}{crit2}
    Suponga que $f$ es continua, positiva y decreciente en un intervalo $[1, \infty)$.  
    Si $a_n = f(n)$, entonces la serie
    \[
        \sum_{i=1}^{\infty} a_n
    \]
    es convergente si y sólo si la integral impropia converge:
    \[
        \int_{1}^{\infty} f(x)\,dx.
    \]
\end{teorema}

\begin{teorema}{Series alternantes}{crit3}
    Una serie alternante es de la forma:
    \[
        \sum_{i=1}^{\infty} (-1)^n b_n = -b_1 + b_2 - b_3 + b_4 - \cdots
    \]

    Condiciones para que sea convergente:
    \[
        \begin{cases}
            b_{n+1} < b_n, \\
            \lim_{n \to \infty} b_n = 0.
        \end{cases}
    \]
\end{teorema}

\begin{teorema}{Álgebra de series}{crit4}
    Si las series $\sum a_n$ y $\sum b_n$ son convergentes, entonces:
    \begin{enumerate}
        \item $\sum (a_n + b_n)$ converge.
        \item $\sum c a_n$ converge.
        \item $\sum a_n b_n$ converge.
        \item $\sum \frac{a_n}{b_n}$ converge.
        \item $\sum (a_n)^n$ converge con $n$ fijo.
    \end{enumerate}
\end{teorema}

\begin{teorema}{Criterio de la comparación}{crit5}
    Sea $a_i, b_i \geq 0$ para todo $i \in \mathbb{N}$.
    \begin{enumerate}
        \item Si $a_i \leq b_i$ y 
        \[
            \sum_{i=1}^{\infty} b_i \quad \text{converge},
        \]
        entonces 
        \[
            \sum_{i=1}^{\infty} a_i \quad \text{converge}.
        \]

        \item Si $a_i \geq b_i$ y 
        \[
            \sum_{i=1}^{\infty} b_i \quad \text{diverge},
        \]
        entonces 
        \[
            \sum_{i=1}^{\infty} a_i \quad \text{diverge}.
        \]

        \item Si $a_i$ y $b_i$ son términos positivos y 
        \[
            \lim_{n \to \infty} \frac{a_n}{b_n} = k, \quad k \in (0,\infty),
        \]
        entonces:
        \begin{enumerate}[label=\roman*)]
            \item Si $\sum a_i$ converge $\;\Rightarrow\;$ $\sum b_i$ converge.
            \item Si $\sum a_i$ diverge $\;\Rightarrow\;$ $\sum b_i$ diverge.
        \end{enumerate}
    \end{enumerate}
\end{teorema}

\begin{teorema}{Series $p$}{crit6}
    Considere la serie
    \[
        \sum_{n=1}^{\infty} \frac{1}{n^p}.
    \]

    Entonces:
    \[
        \begin{cases}
            p > 1 \quad &\Rightarrow \quad \text{converge}, \\
            p \leq 1 \quad &\Rightarrow \quad \text{diverge}.
        \end{cases}
    \]
\end{teorema}

