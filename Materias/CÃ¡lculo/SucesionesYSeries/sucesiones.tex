\documentclass[12pt,a4paper]{article}

% Incluye preámbulo
\usepackage[most]{tcolorbox}
\usepackage{amsmath, amsthm, amssymb}

\geometry{margin=2.5cm}

\tcbuselibrary{theorems}

\newtcbtheorem[number within=section]{definicion}{Definición}%
{colback=blue!5!white, colframe=blue!75!black, fonttitle=\bfseries}{def}
\newtcbtheorem[number within=section]{teorema}{Teorema}%
{colback=green!5!white, colframe=green!75!black, fonttitle=\bfseries}{teo}
\newtcbtheorem[number within=section]{proposicion}{Proposición}%
{colback=yellow!10!white, colframe=yellow!50!black, fonttitle=\bfseries}{prop}

\newcommand{\N}{\mathbb{N}}
\newcommand{\Z}{\mathbb{Z}}
\newcommand{\Q}{\mathbb{Q}}
\newcommand{\R}{\mathbb{R}}
\newcommand{\C}{\mathbb{C}}


% Información del documento
\title{\Huge Sucesiones}
\author{Cálculo 2 - MAT2012 - César Pérez}
\date{Otoño 2025}

\begin{document}

\maketitle

\section{Definiciones}
Muy informalmente podemos decir que una sucesión es una "lista ordenada" de valores númericos, cada uno relacionado con su índice (un número natural), es decir, una sucesión tiene la forma: $a_n = \{a_1, a_2, a_3, a_4, ...\}$, en donde $a_i \in \R$ para $i \in \N$ para formalizar esta idea, introducimos las siguientes definiciones:

\begin{definicion}{Función sobreyectiva}{def1}
    Sea $f: A \to B$. Decimos que $f$ es sobreyectiva si 
    \[
        \forall b \in B, \exists a \in A: f(a) = b
    \]
\end{definicion}

\begin{definicion}{Sucesión}{def2}
    Una sucesión es una función (sobreyectiva) sobre los naturales de la forma $f: \N \to \R$ 
\end{definicion}

\begin{notacion}{}{ntc1}
    Denotamos las sucesiones de las siguientes formas:
    \[ 
        a_n \equiv \{a_n\} \equiv \{a_n\}_{n = 1}^{\infty} \equiv f(n) \equiv (a_n)
    \]
\end{notacion}

El primer gran problema de las sucesiones es encontrar la "regla de asignación" es decir la forma de la función que caracteriza a los terminos $n$ de la sucesión. (Notas de OneNote para los ejemplos). Sin embargo, el problema que se roba el espectaculo es el problema de convergencia de sucesiones. O en otras palabras la existencia del límite: 
    
\begin{definicion}{Límite de una sucesión}{def3}
    Decimos que la sucesión $a_n$ tiene límite L, denotado por
    \[ 
        \lim_{n \to \infty} a_n = L
    \]
    si se da que 
    \[ 
        \forall \varepsilon > 0, \exists N \in \N: \text{para } n > N \text{ se cumple } \lvert a_n - L \rvert < \varepsilon
    \]
\end{definicion}

\begin{definicion}{Sucesiones de Cauchy}{def4}
    Decimos que la sucesión $a_n$ tiene límite L, denotado por
    \[ 
        \lim_{n \to \infty} a_n = L
    \]
    si se da que 
    \[ 
        \forall \varepsilon > 0, \exists N \in \N: \lvert a_n - a_m \rvert < \varepsilon, \forall m,n > N 
    \]
\end{definicion}

También tenemos definiciones para categorizar estas sucesiones, tenemos que principalmente las podemos dividir de la siguente forma:

\begin{definicion}{Sucesiones monotonas}{def5}
    \begin{enumerate}
        \item Decimos que una sucesión es creciente si $a_n \leq a_{n + 1}$
        \item Decimos que una sucesión es decreciente si $a_{n + 1} \leq {a_n}$
        \item Si una sucesión es creciente o decreciente decimos que es monotona
    \end{enumerate}
\end{definicion}

La siguiente definición es importante para un teorema de convergencia de sucesiones:

\begin{definicion}{Acotamiento}{def6}
    Decimos que un conjunto $A$ esta:
    \begin{enumerate}
        \item Acotado superiormente si: $\exists M \in \R: \forall a \in A, a \leq M$
        \item Acotado inferiormente si: $\exists m \in \R: \forall a \in A, m \leq a$
        \item Acotado si: $\exists \mathbf{M} \in \R: \forall a \in A, |a| \leq \mathbf{M}$
    \end{enumerate}
\end{definicion}

\section{Teoremas}

El primer y más importante teorema que podemos vamos a utilizar en esta sección es:

\begin{teorema}{Sucesión de funciones}{thm1}
    Sea la sucesión $a_n$, si:
    \[ 
        \lim_{n \to \infty} f(x) = L \text{ y } f(n) = a_n, \forall n \in \N
    \]
    entonces
    \[ 
        \lim_{n \to \infty} a_n = L
    \]
\end{teorema}

La idea detrás del teorema es definir una función de números reales, la cual sabemos como cálcular los límites al infinito, con la función siendo igual a la sucesión en cada número natural.\\

A continuación se presentan los diferentes teoremas con los que contamos para cálcular la convergencia de sucesiones y el valor a los cuales converge.

\begin{teorema}{Álgebra de sucesiones}{thm2}
    Sea $\{a_n\}$ y $\{b_n\}$ sucesiones con límites existentes, respectivamente cada límite vale $L_1$ y $L_2$ y $c\in\mathbb{R}$. Se cumple:
    \[
    \begin{aligned}
        &\text{Suma/Resta:} & \lim_{n \to \infty} (a_n \pm b_n) &= \lim_{n \to \infty} a_n \pm \lim_{n \to \infty} b_n,\\
        &\text{Producto por escalar:} & \lim_{n \to \infty} (c\,a_n) &= c\,\lim_{n \to \infty} a_n,\\
        &\text{Producto:} & \lim_{n \to \infty} (a_n b_n) &= (\lim_{n \to \infty} a_n)(\lim_{n \to \infty} b_n),\\
        &\text{Cociente:} & \lim_{n \to \infty} \frac{a_n}{b_n} &= \frac{\lim\limits_{n \to \infty} a_n}{\lim\limits_{n \to \infty} b_n}, \quad \lim_{n\to\infty} b_n \neq 0. \\
        &\text{Potencia:} & \lim_{n \to \infty} (a_n)^m &= L_1^m,\\
        &\text{Continuidad ($f$ continua): } & \lim_{n \to \infty}f(a_n) = f(L_1) \\
    \end{aligned}
    \]
\end{teorema}

\begin{teorema}{Criterio del punto fijo}{thm3}
    Sea $(a_n)$ una sucesión, y supongamos que $\lim_{n \to \infty} a_n = L$. Se cumple que:
    \[ 
        \lim_{n \to \infty} a_n = \lim_{n \to \infty} a_{n + 1} = L
    \]
    En particular si $a_{n + 1} = f(a_n)$ con $f$ continua, se da que $L = f(L)$
\end{teorema}

\begin{teorema}{Compresión de sucesiones}{thm4}
    Si $\lim_{n \to \infty} a_n = L = \lim_{n \to \infty} c_n$ y $a_n \leq b_n \leq c_n$ entonces
    \[ 
        \lim_{n \to \infty} b_n = L
    \]
\end{teorema}

\begin{teorema}{Convergencia del absoluto}{thm5}
    Si $\lim_{n \to \infty} \lvert a_n \rvert = L$ entonces $\lim_{n \to \infty} a_n = L$
\end{teorema}

\begin{teorema}{Convergencia de subsucesiones}{thm6}
    Si la sucesión $(a_n)$ es convergente el límite de cualquier subsucesión es convergente
\end{teorema}

Para el siguiente teorema necesitamos \ref{def:def5} y \ref{def:def6}:

\begin{teorema}{Sucesiones acotadas y monotonas}{thm7}
    Sea $(a_n)$ una sucesión real. 
    Si es creciente y acotada superiormente, o decreciente y acotada inferiormente, entonces converge.
\end{teorema}

\end{document}
